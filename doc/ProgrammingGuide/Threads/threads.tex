%
% $Id: threads.tex,v 1.3 2011-05-09 15:43:13 nick.cross Exp $
%
Threads that are created and used by JacORB are described below.

\section*{Long--lived threads}

\subsection*{RequestProcessor}
The RequestProcessor thread invokes servant code when the thread is assigned
a request from the RequestController. This thread invokes firstly the server
request interceptors, then the servant manager, and then the servant code.
Finally, the RequestProcessor invokes interceptors and servant managers and
writes results to the socket when the servant returns the control flow.

The number of RequestProcessor threads which can run is between
{\tt jacorb.poa.thread\_pool\_min} and {\tt jacorb.poa.thread\_pool\_max} times
the number of POAs, or just between those two bounds when
{\tt jacorb.poa.thread\_pool\_shared} is set to ``on''. RequestProcessor threads
will terminate when the POA is destroyed (in other words when the property is
set to ``off'' and when every POA has it's own pool of RequestProcessors) or
when {\tt ORB.shutdown()} is called, subject to the value of the
{\tt jacorb.poa.thread\_pool\_shared} property.

The RequestProcessor thread is implemented in
{\tt org/jacorb/poa/RequestProcessor.java}. Thread instances are pooled
in {\tt org/jacorb/poa/RPPoolManager.java}.

\subsection*{RequestController}
The RequestController assigns requests to RequestProcessors and keeps track of
active requests, object and POA state. The POA state is checked when the
ServerMessageReceptor reads a request from the socket. Request
processing can continue if the POA state is active. However, if the POA is
inactive or if it is being shut down, then the request is rejected.
If the target object is present and not being deactivated, then a
RequestProcessor thread is allocated from the pool and the request is handed over to
the that thread. One RequestController thread is always provided for
each POA: the thread is terminated when the POA is destroyed.

The RequestController thread is implemented in
{\tt org/jacorb/poa/RequestController.java}. A reference to the thread
is retained by {\tt org/jacorb/poa/POA.java}.

\subsection*{ServerSocketListener, SSLServerSocketListener}
These two threads listen on their respective server sockets and accept new
connections. Accepted connections are handed to a thread pool. The
ServerMessageReceptor uses the thread pool to listen on connections for
individual messages.

There can be a maximum of one ServerSocketListener and one
SSLServerSocketListener per ORB, depending on the configuration. These threads
will terminate when {\tt ORB.shutdown()} is called.

The ServerSocketListener and SSLServerSocketListener threads are implemented
in the inner classes {\tt Acceptor} and {\tt SSLAcceptor} in
{\tt org/jacorb/ orb/iiop/IIOPListener.java}: a reference is retained by the
class.

\subsection*{ServerMessageReceptor}
ServerMessageReceptor threads listen on established connections and read new
requests from them. The request's message header is decoded and the POA
name is retrieved from the object key after basic checks are made. The request
is then handed to the POA for scheduling by the RequestController.

The number of ServerMessageReceptor threads is between 0 and the value of
{\tt jacorb.connection.server .max\_receptor\_threads}. This upper bound
also indicates the maximum number of connections that can be serviced
simultaneously. The maximum number of idle threads can be configured using
{\tt jacorb.connection.server.max\_idle\_receptor\_threads}.

ServerMessageReceptor threads terminate when either {\tt ORB.shutdown()} is
called or when the number of idle threads exceeds the maximum specified by
{\tt jacorb.connection.server.max\_idle\_receptor\_threads}.

The ServerMessageReceptor thread is implemented in
{\tt org/jacorb/orb/giop/MessageReceptor.java}: instances are pooled in
{\tt org/jacorb/orb/giop/MessageReceptorPool.java}. Both these classes rely on
and implement interfaces from JacORB's generic thread pool in
{\tt org/jacorb/util/threadpool}.

\subsection*{ClientMessageReceptor}
ClientMessageReceptor threads listen on established connections and read new
replies recieved from them. The request's message header is decoded and the
reply is handed back to the thread that sent the original request after basic
checks are performed. The number of threads which are allowed is between 0 and
the value of {\tt jacorb.connection.client .max\_receptor\_threads}. This upper
bound also indicates the maximum number of connections that can be serviced
simultaneously. The maximum number of idle threads allowed can be set using
{\tt jacorb.connection.client.max\_idle\_receptor\_threads}.

ClientMessageReceptor threads terminate when either {\tt ORB.shutdown()}
is called or when the number of idle threads exceeds the maximum specified by
{\tt jacorb.connection.client.max\_idle\_receptor\_threads}.

This thread is implemented in {\tt org/jacorb/orb/giop/MessageReceptor.java}
and its instances are pooled in
{\tt org/jacorb/orb/giop/MessageReceptorPool.java}. Both these classes rely on
and implement interfaces from JacORB's generic thread pool in
{\tt org/jacorb/util/threadpool}.

\subsection*{BufferManagerReaper}
The BufferManagerReaper thread ensures that the extra-large buffer cache entry
will not live longer than the time specified by
{\tt jacorb.bufferManagerMaxFlush}. The BufferManagerReaper thread exits when
{\tt ORB.shutdown()} is called.

This thread is implemented as inner class {\tt Reaper} in {\tt
org/jacorb/orb/BufferManager.java} and a reference is kept by the class.

\subsection*{AOMRemoval}
This thread is used to ensure that calls to deactivate\_object return
immediately. When an object is removed it is placed on a java.util.concurrent.LinkedBlockingQueue
which this thread processes to finish deactivation of the objects. This thread is a daemon thread.

\section*{Short--lived threads}

\subsection*{POAChangeToActive}
The POAChangeToActive thread asynchronously sets the state of those POAs
controlled by a POAManager to active. A new thread will be created whenever
{\tt POAManager.activate()} is called. The thread terminates when all POAs
have been activated.

The POAChangeToActive thread is implemented as an anonymous inner class in
{\tt org/jacorb/poa/POAManager.java}.

\subsection*{POAChangeToInactive}
The POAChangeToInactive thread asynchronously sets the state of the POAs
controlled by a POAManager to inactive. A new thread will be created whenever
{\tt POAManager.deactivate()} is called. The thread terminates when all
POAs have been deactivated.

The POAChangeToInactive thread is implemented as an anonymous inner class in
{\tt org/jacorb/poa/POAManager.java}.

\subsection*{POAChangeToDiscarding}
The POAChangeToDiscarding thread asynchronously sets the state of those POAs
controlled by a POAManager to discarding. A new thread is created whenever
{\tt POAManager.discard\_requests()} is called. This thread terminates when
all POAs have been set to discarding.

The POAChangeToDiscarding thread is implemented as an anonymous inner class in
{\tt org/jacorb/poa/POAManager.java}.

\subsection*{POAChangeToHolding}
The POAChangeToHolding thread asynchronously sets the state of those POAs
controlled by a POAManager to holding. A new thread is created whenever
{\tt POAManager.hold\_requests()} is called. This thread when all POAs have
been set to holding.

The POAChangeToHolding thread is implemented as an anonymous inner class in
{\tt org/jacorb/poa/POAManager.java}.

\subsection*{POADestructor}
The POADestructor thread allows asynchronous destruction of a POA. This thread
initially synchronizes with the RequestController which waits until all active
requests have been finished. Then, all unprocessed requests are discarded
by the RequestController thread and destruction of the POA is completed. The
thread will then exit.

One POADestructor thread is created whenever {\tt POA.destroy()} is called.
Note that destroying a POA will destroy all child POAs. Accordingly, there will
be many threads as there are POAs, including child POAs, which are to be
destroyed.

The POADestructor thread is implemented as an anonymous inner class in
{\tt org/jacorb/poa/POA.java}.

\subsection*{PassToTransport}
The PassToTransport thread is created and performs the network send task
whenever a request is sent with the sync scope set to {\tt SYNC\_NONE}. The
thread exits when it is finished sending and allows the client thread to return
immediately.

The PassToTransport thread is implemented as an anonymous inner class in
{\tt org/jacorb/orb/Delegate.java}.

\subsection*{ReplyReceiverTimer}
The ReplyReceiverTimer thread manages the termination point for reply timeouts.
The thread is created for each anticipated reply when the ReplyEndTime
policy is set. The thread exits when the timeout expires or the
anticipated reply is received before timeout expires.

The ReplyReceiverTimer thread is implemented as inner class {\tt Timer} in
{\tt org/jacorb/orb/ReplyReceiver.java} and a reference is kept by the class.

\subsection*{SocketConnectorThread}
The SocketConnectorThread thread connects to the socket for every new
connection to the server when {\tt jacorb.connection.client.connect\_timeout}
is set to a value greater than zero (0). The SocketConnectorThread thread
provides timeout control which is not available in older JDK versions

The thread exits when either the connection is successfully established or
when the timeout expires.

The ReplyReceiverTimer thread is implemented as an anonymous inner class in
{\tt org/jacorb/orb/ClientIIOP\-Connection.java}.
