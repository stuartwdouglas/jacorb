%
% Contents???
% Properties
% Corbaloc
% Demo example


JacORB has an implementation of MIOP written as an ETF plugin. This
conforms to version 03-01-11 of the Unreliable Multicast Inter-ORB
Protocol specification.


\section{Enabling the MIOP Transport}
In order to enable the ETF transport plugin the following configuration
properties must be altered.
\begin{verbatim}
jacorb.transport.factories
jacorb.transport.client.selector
\end{verbatim}
By default these properties are configured to use the IIOP transport. For example
to select both IIOP and MIOP transports:
\begin{verbatim}
jacorb.transport.factories=org.jacorb.orb.iiop.IIOPFactories,
                           org.jacorb.orb.miop.MIOPFactories
jacorb.transport.client.selector=
                           org.jacorb.orb.miop.MIOPProfileSelector
\end{verbatim}


\section{Configuring the MIOP Transport}
\label{miopConfig}
A number of extra configuration properties have been added for the transport.



\begin{small}
\begin{longtable}{|p{5cm}|p{9cm}|p{2cm}|}
\caption{MIOP Configuration}\\
\hline
~ \hfill \textbf {Property} \hfill ~ & ~ \hfill \textbf {Description} \hfill ~ & ~ \hfill \textbf {Type} \hfill ~ \endhead
\hline
\verb"jacorb.miop.timeout" & Timeout used in MIOP requests. Default is 100. &
 integer \\
\hline
\verb"jacorb.miop.time_to_"
\verb"live" & TTL used for multicast UDP packets. Default
is 5 seconds. & integer \\
\hline
\verb"jacorb.miop.incomplete_"
\verb"messages_threshold" & Maximum number of incomplete messages allowed.
Default 5. & integer \\
\hline
\verb"jacorb.miop.message_"
\verb"completion_timeout" & Timeout for packet collection to be completed.
Default 500ms. & integer \\
\hline
\verb"jacorb.miop.packet_"
\verb"max_size" & This is the maximum size of the frame buffer. This defaults
to 1500 bytes which is the typical value for most network interfaces. From this
the IP, UDP and UMIOP headers will be deducted which will leave 1412 bytes for
the MIOP packet. & integer \\
\hline
\end{longtable}
\end{small}



\section{MIOP Example}

A new demo has been included within {\tt <JacORB>/demo/miop}. This section will
describe how to run this demo including its use of MIOP corbaloc strings.

Assuming the developer has installed Ant {\small version 1.7.1} or above then
the example may be compiled by typing {\tt ant} within the {\tt <JacORB>/demo/miop}
directory. The classes will be compiled to {\tt <JacORB>/classes} which may
need to be added to the classpath.


To run the server:
\begin{small}
\begin{verbatim}
jaco demo.miop.Server
\end{verbatim}
\end{small}

To run the client:
\begin{small}
\begin{verbatim}
jaco demo.miop.Client
\end{verbatim}
\end{small}

This is the simplest configuration and will simply send two oneway requests via
UDP to the server. By default the Server will write out a miop.ior file
containing the following corbaloc:

\begin{verbatim}
corbaloc:miop:1.0@1.0-TestDomain-1/224.1.239.2:1234;
         iiop:1.2@10.1.0.4:38148/4222541922/%00%16%0F%205=%25%02%01I%0C
\end{verbatim}
The Group IIOP Profile key string will not remain constant. The server takes a
single optional argument:
\begin{tabbing}
XX \= XXXXXXXXXXXXX \= XX \kill
\>  -noGroupProfile \>  Don't write IIOP Group Profile
request.\\
\end{tabbing}
This will create a corbaloc as shown below and is useful for interoperating
with ORBs that do not support the Group Profile.
\begin{verbatim}
corbaloc:miop:1.0@1.0-TestDomain-1/224.1.239.2:1234
\end{verbatim}


The Client takes two optional arguments:
\begin{tabbing}
XX \= XXXXXXXXXXXXX \= XX \kill
\>  -fragment \>  Trigger fragmentation by sending a larger request.\\
\>  [IOR|Corbaloc] \> Don't use miop.ior but this supplied IOR or Corbaloc.\\
\end{tabbing}

The second optional argument is useful if interoperating with another ORB.

\subsection{Two way requests and MIOP}
The demo client does an unchecked\_narrow on the supplied corbaloc/URL. This is
because a MIOP URL does not normally support a two-way is\_a request unless a
Group IIOP profile has also been encoded into the corbaloc. By default the
JacORB demo server will create the Group IIOP profile as well:

\begin{verbatim}
corbaloc:miop:1.0@1.0-TestDomain-1/224.1.239.2:1234;
         iiop:1.0@10.1.0.4:36840/7150661784/%00%16%0F%1B*@2%02,%1A
\end{verbatim}

It is not guaranteed that other ORBs (e.g. TAO) will create the Group IIOP
profile.
