%
% $Id: configuration.tex,v 1.53 2010-11-17 12:37:24 alexander.bykov Exp $
%

This chapter explains the general mechanism for configuring JacORB
and lists all configuration properties. Note that as JacORB's
configuration has been updated it is recommended to use the new
jacorb.properties file supplied with this version.

\section{Properties}

JacORB has a number of configuration options which can be set as Java
properties. There are three options for setting properties:

\begin{itemize}
\item in properties files
\item as command line properties, and
\item as properties passed as arguments to ORB.init() in the code of your
  applications.
\end{itemize}

In the case of a single JVM with multiple ORB instances, it may be
required to either share configuration options between ORBs, or to
separate the individual configurations from each other. We explain how
properties can be set for sharing or for individual ORB instances.

\subsection{Properties files}

JacORB looks for a few standard properties files, a common file called
{\tt orb.properties}, and an ORB-specific file called {\tt
  <orbid>.properties}, where {\tt <orbid>} is the name of an ORB
instance that was explicitly configured. Moreover, JacORB can load
custom properties files from arbitrary locations. We explain each of
these files in turn.

\subsubsection{The common properties file}

The reason for having a common properties file is that a single JacORB
installation may be shared by a number of users with a set of common
default properties. These may be refined by users in their own
properties files but still provide reasonable defaults for the
environment. Note that it is not required to have a common properties
file as all configuration options can also be set in other files, on
the commandline or in the code.

JacORB looks for the common properties file {\tt orb.properties} in
the following places:

\begin{enumerate}
\item in the {\tt lib} directory of the JDK installation. (The JDK's
  home directory denoted by the system property "java.home").
\item in the user home directory. (This is denoted by the system
    property "user.home". On Windows, this is
    {\verb+c:\documents\username+}, on Unixes it's {\verb+~user+}. If
    in doubt where your home directory is, write a small Java programm that
  prints out this property.
\item on the class path.
\end{enumerate}

The common properties file is searched in the order presented above,
so you may actually be loading multiple files of this name. If a
properties file is found it is loaded, and any property values defined
in this file will override values of the same property that were
loaded earlier. Loading properties files from the classpath is useful
when distributing applications packaged in JAR files.

\subsubsection{The ORB properties file}

Having ORB-specific properties files is necessary when multiple ORB
instances live in the same process, but need to have separate
configurations, e.g., some ORBs use SSL and others don't, or some ORBs
need to listen on separate but predefined ports. To let colocated ORBs
use and retrieve separate configurations, JacORB provides a lookup
mechanisms based on a specific property, the {\tt ORBid} property. The
default value for the ORBid is {\tt jacorb}, ie. is the ORBid is not
explicitly set anywhere, it defaults to {\tt jacorb}. Note that this
ORBid is reserved, ie., you cannot explicitly set your ORBid to this
value. To use different configurations for different ORBs, you simply
pass different ORBid values to your ORBs.

JacORB looks for ORB properties files in these places:

\begin{enumerate}
\item {\tt jacorb.config.dir/etc/orbid.properties.}, if that exists, or
\item {\tt jacorb.home/etc/orbid.properties.}, or
\item the current directory ({\tt './orbid.properties.'})
\item on the class path.
\end{enumerate}

The {\tt jacorb.config.dir} and {\tt jacorb.home} properties must be
set for JacORB to be able to use a preconfigured configuration
directory. The {\tt jacorb.home} property defaults to {\tt ``.''}, if
unset. Setting these properties can be done in the {\tt
  orb.properties} file, or by passing a property in on the
commandline, like this:

\cmdline{jaco -Djacorb.config.dir=c:/ -DORBid=example test.Example}

This commandline causes JacORB to look for a file called {\tt
  example.properties} in {\tt c:/etc}. If the {\tt -DORBid=example}
had been ommitted, the name of the ORB properties file that JacORB
would try to load would have been {\tt jacorb.properties}, because
that is the default value for the ORBid. A good starting point is to
have a common properties file that sets the {\tt jacorb.config.dir}
property, and then have put a {\tt jacorb.properties} file in that
directory.

Note, however, that the added flexibility of using multiple
configuration files may lead to individual properties defined in
multiple files. You must know the order in which your configuration
files are loaded to avoid confusion over property settings not having
the expected effect!

\subsubsection{Custom properties files}

In addition to the standard JacORB properties files, a {\em custom
  properties file} can be loaded by passing the name of that
properties files the {\tt custom.props} property to JacORB. This can
be handy for application-specific settings that you want to distribute
with your code.

The value of this property is the path to a properties file, which
contains the properties you want to load. As an example, imagine that
you usually use plain TCP/IP connections, but in some cases want to
use SSL (see section \ref{ch:SSL}). The different ways of achieving
this are

\begin{itemize}
\item Use just one properties file, but you will have to edit that
  file if you want to switch between SSL and
plaintext connections.
\item Use commandline properties exclusively (cf. below), which may lead to very long
commands
\item Use a command property file for all applications and different
  custom properties files for each application.
\end{itemize}

For example, you could start a JacORB program like this:

\cmdline{jaco -Dcustom.props=c:/tmp/ns.props org.jacorb.naming.NameServer}

In addition to loading any standard properties files found in the
places listed above, JacORB will now also load configuration
properties from the file {\tt c:/tmp/ns.props}, but this last file
will be loaded after the default properties files and its values will
thus take precedence over earlier settings.

\subsection{Command-line properties}

In the same way as the {\tt custom.props} property in the example
above, arbitrary other Java properties can be passed to JacORB
programs using the {\tt -D<prop name>=<prop value>} command line
syntax for the {\tt java} interpreter, but can be used in the same way
with the {\tt jaco} script. Note that the properties must precede
the class name on the command line. For example to override the
ORB initial references for NameService the following may be used:

\small{
\begin{verbatim}
    jaco -DORBInitRef.NameService=file:///usr/users/...../NameService.ior
        Server
\end{verbatim}

The ORB configuration mechanism will give configuration properties
passed in this way precedence over property values found in
configuration files.

Anything that follows after the class name is interpreted (by {\tt java}) as a
command line argument to the class and will be visible in the {\tt args}
parameter of the classes main method. For example

\small{
\begin{verbatim}
    jaco Server
        -ORBInitRef.NameService=file:///usr/users/..../NameService.ior
\end{verbatim}

\subsection{Arguments to ORB.init()}

For more application--specific properties, you can pass a {\tt
 java.util.Properties} object to {\tt ORB.init()} during application
initialization. Properties set this way will override properties set
by a properties file. The following code snippet demonstrates how to
pass in a {\tt Properties} object ({\tt args} is the String array
containing command line arguments):

\small{
\begin{verbatim}
    java.util.Properties props = new java.util.Properties();
    props.setProperty("jacorb.implname","StandardNS");
    org.omg.CORBA.ORB orb = org.omg.CORBA.ORB.init(args, props);
\end{verbatim}
}

\section{Common Configuration Options}

We are now ready to have a look at the most basic JacORB configuration
properties. As a starting point, you should look at the file {\tt
  /etc/jacorb\_properties.template}, which you can adapt to your own
needs (e.g. renaming to jacorb.properties or orb.properties as required).

\subsection{Initial references}

Initial references are object references that are available to CORBA
application through the bootstrap {\tt
  orb.resolve\_initial\_service()} API call. This call takes a string
argument as the name of an initial reference and returns a CORBA
object reference, e.g., to the initial name service.

\renewcommand{\baselinestretch}{0.9}
\small{
\begin{verbatim}
########################################
#                                      #
#   Initial references configuration   #
#                                      #
########################################

#
# URLs where IORs are stored (used in orb.resolve_initial_service())
# DO EDIT these! (Only those that you are planning to use,
# of course ;-).
#
# The ORBInitRef references are created on ORB startup time. In the
# cases of the services themselves, this may lead to exceptions being
# displayed (because the services aren't up yet). These exceptions
# are handled properly and cause no harm!

#ORBInitRef.NameService=corbaloc::160.45.110.41:38693/StandardNS/NameServer-POA/_root
#ORBInitRef.NameService=file://c:/NS_Ref
ORBInitRef.NameService=http://www.x.y.z/~user/NS_Ref
#ORBInitRef.TradingService=http://www.x.y.z/~user/TraderRef
\end{verbatim}
}
\renewcommand{\baselinestretch}{1.0}
\small\normalsize

The  string value  for  {\tt
ORBInitRef.NameService} is  a URL  for a resource  used to set  up the
JacORB name  server. This URL  will be used  by the ORB to  locate the
file  used to  store  the  name server's  object  reference (see  also
chapter \ref{ch:naming}).

\subsection{Logging}

JacORB writes logging information through SLF4J, which is a logging
facade that can interface with arbitary backend logging systems such
as Log4J, JCL, or JDK logging.  To switch to a different logging
system, all that needs to be done is to put a different library on the
classpath.

JacORB does not usually attempt to configure the external logging
system.  That means it is left to you to provide a configuration for
it, for example to set the log verbosity or to configure log file
names.  This is done in a way that is specific to the particular
logging system used, e.g. via property files.  As an added
convenience, there is also a hook in JacORB that allows you to make
some settings in the external logging system based on the
configuration of JacORB itself, for example to choose log file names
based on the implementation name of the server.

The default logging system selected in the JacORB distribution is JDK
logging.  This is because it is already present in the JDK, and JacORB
therefore does not need to ship any other library.  A sample
configuration file for JDK logging is provided, which makes it easy to
specify log file names, log rotation parameters, and the like.

\subsubsection{Logging Conventions}

JacORB logs to a hierarchy of loggers with a root logger named {\tt
  jacorb}.  All sub-loggers have names that start with this prefix,
e.g. {\tt jacorb.orb}, {\tt jacorb.config}, and so on.  Settings that
apply to the root logger usually also apply to all loggers below in
the hierarchy (depending on the actual logging system used).

It is possible to split the logging hierarchy based on the
implementation name of the ORB instance.  Different ORBs will then log
to different loggers, which can be configured independently.  To do
this, set the property {\tt jacorb.log.split\_on\_implname} to {\tt
  true}.  Then, if the property {\tt jacorb.implname} is set for an
ORB instance, the loggers for that ORB all start with the prefix {\tt
  jacorb.$implname$.}, rather than just {\tt jacorb.}.

SLF4J defines five different logging levels: \emph{error},
\emph{warn}, \emph{info}, \emph{debug}, and \emph{trace}.  These
levels are mapped to the log levels of the underlying logging system.
For JDK logging, the mapping is \emph{error} $\rightarrow$ {\tt
  SEVERE}, \emph{warn} $\rightarrow$ {\tt WARNING}, \emph{info}
$\rightarrow$ {\tt INFO}, \emph{debug} $\rightarrow$ {\tt FINE}, and
\emph{trace} $\rightarrow$ {\tt FINEST}.  In JacORB's code, the
SLF4J log levels are used according to the following conventions:

\begin{description}
\item[error] Events which suggest that there is a bug in JacORB or in
  user code.  This includes, but is not limited to, ``fatal errors''
  which will lead to termination of the program.

\item[warn] Events that demand attention, but which are handled
  properly according to the CORBA spec.  For example, abnormal
  termination of a connection, reaching of a resource limit (queue
  full), and the like.

\item[info] Starting and stopping of subsystems, establishing and
  closing of connections, registering objects with a POA.

\item[debug] Information that might be needed for finding bugs in
  JacORB or user code.  Anything that relates to the normal processing
  path of individual messages.

\item[trace] Not used in JacORB (and discouraged by the SLF4J team).
\end{description}

\subsubsection{Configuration of JDK Logging}

Since JDK logging is the default in the JacORB distribution, we
provide some shortcuts to achieve a reasonable logging configuration
with it easily.  If any of the two properties {\tt
  jacorb.log.default.verbosity} or {\tt jacorb.logfile} is set, then
JacORB configures JDK logging at startup to match these values.
(These properties were retained from previous JacORB versions where
JacORB configured all logging itself.)

The property {\tt jacorb.log.default.verbosity} specifies the level at
which messages are logged.  The value is a number from 0 to 4, where 0
means no logging (off), 1 means only $error$ messages, 2 means $warn$
messages, 3 means $info$ messages, and 4 means $debug$ messages (higher
levels also include lower levels).

The property {\tt jacorb.logfile} specifies the name of a file to
write the log to.  If this property is not set, then logging goes to
the console.  You may include the string ``\$implname'' in the value
of this property; this is replaced with the value of the property
{\tt jacorb.implname} if that is set, or ``jacorb'' otherwise.

The default formatting of JDK logs is quite verbose, using two lines
of output for every log entry.  We have provided a {\tt LogFormatter}
class that gives a more succinct output.  This class is named {\tt
  org.jacorb.config.JacORBLogFormatter} and is used whenever JacORB
configures the logging system itself.

For any more sophisticated configuration, such as using log file
rotation or specifying different log levels for some of the loggers,
you need to configure JDK logging directly.  To do this, make sure
that the properties {\tt jacorb.log.default.verbosity} and {\tt
  jacorb.logfile} are \emph{not} set to any value (because otherwise
JacORB will interfere with your configuration).  Then, prepare a
configuration file such as the one found in {\tt
  JACORB\_HOME/etc/logging\_properties.template} with your settings.
The name of this file needs to be passed to the JVM at startup, for
example by providing the following command line option:

\begin{verbatim}
-Djava.util.logging.config.file=/path/to/config-file
\end{verbatim}

\subsubsection{Using another Logging System}

The SLF4J facade is implemented in two Java libraries, a generic one
and a backend-specific one.  The generic library is named {\tt
  slf4j-api-1.5.10.jar} (or any other version), and the
backend-specific library is named {\tt slf4j-jdk14-1.5.10.jar} (for JDK
logging), or {\tt slf4j-log4j-1.5.6.jar} for Log4J, and the like.  To
switch to a different logging system, the backend-specific library of
SLF4J needs to be replaced, and the implementation library for that
backend needs to be added as well.  So for example, to use Log4J, the
following libraries need to be on the classpath: {\tt
  slf4j-api-1.5.10.jar}, {\tt slf4j-log4j-1.5.10.jar}, and {\tt
  log4j-1.2.15.jar}.  The JacORB distribution only ships with the
generic SLF4J library and the JDK adapter library.  Other libraries
need to be downloaded from the SLF4J site.

When using another logging system besides JDK logging, JacORB does not
attempt to configure log verbosity or log file names by itself, as
described in the previous section.  This means that features such as
choosing log file names based on implementation names are not
available for other logging systems.  You can however configure this
explicitly, for example by using the split-on-implname feature
described above.  For more sophisticated needs, it is also possible to
provide a {\tt LoggingInitializer} class, which is a hook provided by
JacORB to allow configuration of a logging system based on the JacORB
configuration.  The class needs to extend the class {\tt
  org.jacorb.config.LoggingInitializer} and the name of the class
needs to be specified in the property {\tt jacorb.log.initializer}.

\subsubsection{POA Monitor}

The  {\tt jacorb.poa.monitoring} property  determines whether  the POA
should bring up a monitoring GUI  for servers that let you examine the
dynamic behavior of  your POA, e.g.  how long  the request queue gets
and whether your thread pool is  big enough.  Also, this tool lets you
change the  state of a POA,  e.g. from {\it active}  to {\it holding}.
Please see chapter \ref{ch:POA} on the POA for more details.

\section{Configuration Properties}

A comprehensive listing and description of the properties which are used
to configure JacORB are given in the following tables.

\begin{small}
\begin{longtable}{|p{5cm}|p{7.5cm}|p{1.5cm}|p{1.5cm}|}
\caption{ORB Configuration}\\
\hline
~ \hfill \textbf {Property} \hfill ~ & ~ \hfill \textbf {Description}
\hfill ~ & ~ \hfill \textbf {Type} \hfill ~ & \hfill \textbf{Default} \endhead
\hline
\verb"ORBInitRef.<service>" & Properties of this form configure
initial service objects which can be resolved via the ORB
resolve\_initial\_references. A variety of URL formats are
supported. & URL & unset \\
\hline
\verb"org.omg.PortableInterc"
\verb"eptor.ORBInitializerCl"
\verb"ass.<name>" & A portable interceptor initializer class
instantiated at ORB creation. & class & unset \\
\hline
\verb"jacorb.orb.objectKeyMa"
\verb"p.<name>" & Maps an object key to an arbitrary string thereby
enabling better readability for corbaloc URLs. & string & \\
\hline

\verb"jacorb.giop_minor_vers"
\verb"ion" & The GIOP minor version number to use for newly created
IORs & integer & 2 \\
\hline
\verb"jacorb.retries" & Number of retries if connection cannot
directly be established & integer & 5 \\
\hline
\verb"jacorb.retry_interval" & Time in milliseconds to wait between
retries & millisec. & 500 \\
\hline
\verb"jacorb.buffermanager.f"
\verb"actory" & This parameter allow to define buffer manager
factory. Here are 3 options already implemented:
\begin{enumerate}
\item org.jacorb.orb.DefaultBufferManagerFactory that will create
    default buffer manager implementation
\item org.jacorb.orb.JDK15BufferManagerFactory that uses JDK 1.5 (or
    above) buffer manager implementation based on the soft references
    (java.lang.ref.SoftReference).
\item org.jacorb.orb.NonCachingBufferManagerFactory that uses simple
    buffer manager implementation without any caching.
\end{enumerate}
Also, custom-made buffer manager factories allowed. They must implement
the org.jacorb.orb.BufferManagerFactory interface.
& class & org.jacorb.orb.DefaultBufferManagerFactory \\
\hline
\verb"jacorb.maxManagedBufSi"
\verb"ze" & This is NOT the maximum buffer size that can be used, but
just the largest size of buffers that will be kept and managed. This
value will be added to an internal constant of 5, so the real value in
bytes is 2** (5 + maxManagedBufSize - 1). You only need to increase
this value if you are dealing with LOTS of LARGE data structures. You
may decrease it to make the buffer manager release large buffers
immediately rather than keeping them for later reuse & integer & 18 \\
\hline
\verb"jacorb.bufferManagerFl"
\verb"ushMax" & Whether to use an additional unlimited size buffer
cache for CDROutputStreams. If -1 then off, if zero then this is
feature is enabled, if greater than zero then it is enabled and
flushed every x seconds & integer & -1 \\
\hline
\verb"jacorb.deferredArrayQu"
\verb"eue" & JacORB will delay internally transferring bytes to the stream;
this is the size of this internal queue. Size in k. & integer & 8. \\
\hline
\verb"jacorb.connection.del"
\verb"ay_close" & Normally, a jacorb server will close the TCP/IP connection
right after sending a CloseConnection message. However, it may occasionally
happen that the client sends a message into the closed connection because
it hasn't handled the CloseConnection yet. To avoid this situation, closing
of the TCP/IP connection can be delayed (Delay time is controlled by
jacorb.connection.timeout\_after\_closeconnection specified in msecs) &
boolean & off \\
\hline
\verb"jacorb.connection.cli"
\verb"ent.connect_timeout" & Initial timeout for establishing a connection.
 & millisec & 90000 \\
\hline
\verb"jacorb.connection.clie"
\verb"nt.pending_reply_timeo"
\verb"ut" &  Wait the specified number of msecs for a reply to a
request. If exceeded, a org.omg.CORBA.TIMEOUT exception will be
thrown. Not set by default & millisec. & 0  \\
\hline
\verb"jacorb.connection.clie"
\verb"nt.idle_timeout" & Client-side timeout. This is set to non-zero in order
to close the connection after specified number of milliseconds idle time. Only
connections that don't have pending messages are closed, unless
jacorb.connection.client.timeout\_ignores\_pending\_messages is turned on. &
millisec. & unset \\
\hline
\verb"jacorb.connection.clie"
\verb"nt.timeout_ignores_pen"
\verb"ding_messages" & Controls if client-side idle timeouts take care of
pending messages or not. If "on", the connection is closed regardless of any
pending messages, and all pending messages are cancelled (resulting in a {\tt
COMM\_FAILURE}, unless jacorb.connection.client.retry\_on\_failure is turned
on).& boolean & off \\
\hline
\verb"jacorb.connection.clie"
\verb"nt.retry_on_failure" & Controls if network failures on existing connections
should yield a COMM\_FAILURE or should trigger a remarshaling
of all pending messages. Note that this should only be used with idempotent
operations because the client side ORB has no way of knowing the processing
state of the lost request on the server. & boolean & \\
\hline
\verb"jacorb.connection.serv"
\verb"er.timeout" & Maximum time in milliseconds that a server keeps a
connection open if nothing happens & millisec. & unset \\
\hline
\verb"jacorb.connection.serv"
\verb"er.keepalive" & Enable SO\_KEEPALIVE on server sockets. If the OS
keepalive detects a TCP/IP connection to be broken, the effect is the same as
if the TCP/IP connection has been closed gracefully. & boolean & false \\

\hline
\verb"jacorb.connection.clie"
\verb"nt.keepalive" & Enable SO\_KEEPALIVE on client sockets. If the OS
keepalive detects a TCP/IP connection to be broken, the effect is the same as
if the TCP/IP connection has been closed gracefully.All pending replies will
receive a {\tt COMM\_FAILURE}. & boolean & false \\
\hline
\verb"jacorb.connection.max"
\verb"_server_connections" & This property sets the
  maximum number of TCP/IP connections that will be listened on by the
  server--side ORB. Only effective in conjunction with the other connection
  management properties. Please see \ref{connection_management}.& integer &
  unlimited \\

\hline
\verb"jacorb.connection.wait"
\verb"_for_idle_interval" & This property sets the
  interval to wait until the next try is made to find an idle connection to
  close. Only effective in conjunction with the other connection management
  properties. Please see \ref{connection_management}. & millisec & 500\\

\hline
\verb"jacorb.listener.server"
\verb"_socket_timeout" & Sets a timeout on the (SSL) server socket. This is a
workaround for JDK 1.3 on linux where a thread blocked on \verb"accept()"
isn't notified when closing that socket. Default is 0, i.e.~off. See Java bug
\#4344135. NOTE: This is only useful in conjunction with the SI\&C SSL socket
factories. & millisec & 0\\
\hline
\verb"jacorb.connection.sele"
\verb"ction_strategy_class" & This property sets
  the {\tt Selection\-Strategy}. Only effective in conjunction with the other
  connection management  properties. Please see \ref{connection_management}. &
  class & \\
\hline
\verb"jacorb.connection.stat"
\verb"istics_provider_class" & This property sets
  the {\tt Statistics\-Provider}. Only effective in conjunction with the other
  connection management properties. Please see \ref{connection_management}. &
  class & \\

\hline
\verb"jacorb.connection.del"
\verb"ay_close" & This property controls the behaviour after sending a GIOP
CloseConnection messsage. If set to ``on'', the TCP/IP connection won't be
closed directly. Instead, it is waited for the client to do so
first. Please see \ref{connection_management}. & boolean & off \\
\hline

\verb"jacorb.listener.server"
\verb"_socket_timeout" & Sets a timeout on the (SSL) server socket. This is a
workaround for JDK 1.3 on linux where a thread blocked on \verb"accept()"
isn't notified when closing that socket. Default is 0, i.e.~off. See Java bug
\#4344135. NOTE: This is only useful in conjunction with the SI\&C SSL socket
factories. & millisec & 0\\
\hline

\verb"jacorb.transport.facto"
\verb"ries" & This property controls which transport plug-ins are
available to the ORB.  The value is a list of classes that implement the ETF
{\tt Factories} interface.
& comma-separated list of classes & \\
\hline
\verb"jacorb.transport.serve"
\verb"r.listeners" & Controls which transports should be offered by
JacORB on the server side.  The value is a list of numeric profile
tags for each transport that should be available on the server side.
& comma-separated list of integers & \\
\hline
\verb"jacorb.transport.clien"
\verb"t.selector" & Name of a class that selects the transport profile
to use for communication on the client side.  The value is the fully
qualified name of a class that implements {\tt
  org.jacorb.orb.ProfileSelector}.
& class & \\
\hline
\verb"jacorb.reference_cachi"
\verb"ng" & Whether or not JacORB caches objects references & boolean & unset  \\
\hline
\verb"jacorb.hashtable_class" & The following property specifies the
class which is used for reference caching. WeakHashtable uses
WeakReferences, so entries get garbage collected if only the Hashtable
has a reference to them. This is useful if you have many references to
short-living non-persistent CORBA objects. It is only available for
java 1.2 and above. On the other hand the standard Hashtable keeps the
references until they are explicitly deleted by calling
\_release(). This is useful for persistent and long-living CORBA
objects & class & Hashtable \\
\hline
\verb"jacorb.use_bom" & Use GIOP 1.2 byte order markers, since CORBA
2.4-5 & boolean & off  \\
\hline
\verb"jacorb.giop.add_1_0_pr"
\verb"ofiles" & Add additional IIOP 1.0 profiles even if using IIOP
1.2 & boolean & off \\
\hline
\verb"jacorb.dns.enable" & Use DNS names in IORs, rather than numeric
IP addresses & boolean & off \\
\hline
\verb"jacorb.dns.eager\_resolve" & resolve DNS names in IORs eagerly & boolean & on \\
\hline
\verb"jacorb.compactTypecode"
\verb"s" & Whether to send compact typecodes. Options are 0 (off), on (full compaction of all optional parameters) & boolean & on \\
\hline
\verb"jacorb.cacheTypecodes" & Whether to cache read
typecodes  & boolean & off \\
\hline
\verb"jacorb.cachePoaNames" & Whether to cache poa names as an optimisation
to save reparsing portions of the object key& boolean & off \\
\hline
\verb"jacorb.orb_initializer"
\verb".fail_on_error" & Control, if failing ORBInitializers should make the
complete {\tt ORB.init()} fail. & boolean & off \\
\hline
\verb"jacorb.acceptor_"
\verb"exception_listener"
\verb"_class" & A class implementing interface {\tt
  org.jacorb.orb.listener.AcceptorException\-Listener}. The implementation
will be notified of any exception caught by the thread doing the {\tt
  ServerSocket.accept()} and has the chance of taking appropriate action,
e.g. shutting down the ORB. The default implementation will shutdown the ORB
on all Errors and SSLExceptions. & String (classname) & org.jacorb
.orb.listener.DefaultAcceptorExceptionListener \\
\hline
\verb"jacorb.interop.indirec"
\verb"tion_encoding_disable" & Turn off indirection encoding for
repeated typecodes. This fixes interoperability with certain broken
ORB's eg. Orbix 2000 & boolean & off \\
\hline
\verb"jacorb.interop.comet" & Enable additional buffer length checking
and adjustment for interoperability with Comet CORBA/COM bridge which
can incorrectly encode buffer lengths & boolean & off
\\
\hline
\verb"jacorb.interop.lax_"
\verb"boolean_encoding" & Treat any non zero CDR encoded boolean value
as true (strictly should be 1 not non zero). This is useful for ORBs such
as VisiBroker and ORBacus & boolean & off \\
\hline
\verb"org.omg.PortableInterc"
\verb"eptor.ORBInitializerCl"
\verb"ass.bidir_init" & This portable interceptor must be configured
to support bi-directional GIOP & class & unset \\
\hline
\verb"jacorb.ior_proxy_addre"
\verb"ss" & Used to supply an alternative
endpoint in locally created object
references. This is intended for servers that export IORs for access
from outside a firewall.  The general form of the value is {\tt
  <protocol>://<address>}. The protocol name in the value must match
the protocol(s) used by the server. For example: {\tt
  iiop://myhost:1234}. The given address is inserted into every IOR
that the local ORB produces, without any check whether the address is
valid, except that the protocol must be supported by the ORB, and the
address must be parsable for that protocol. This property supercedes
jacorb.ior\_proxy\_host and jacorb.ior\_proxy\_port.
& string & unset \\
\hline
\verb"jacorb.ior_proxy_host" & The properties jacorb.ior\_proxy\_host and
jacorb.ior\_proxy\_port have been superceded by
jacorb.ior\_proxy\_address (see above), which is a
protocol-independent way of specifying endpoint addresses.  The
host/port properties are still recognized, but if
jacorb.ior\_proxy\_address is specified, it overrides these properties.
Note that the value that ends up in the IOR also
is affected by the setting of the property jacorb.dns.enable.
& node & unset \\
\hline
\verb"jacorb.ior_proxy_port" & See jacorb.ior\_proxy\_host and
jacorb.ior\_proxy\_address above &
port & unset \\
\hline
\verb"OAAddress" & Used to supply an explicit listener protocol and
address for servers. The general form of the value is
{\tt <protocol>://<address>}. The protocol name must match the
protocol(s) used by the server. For example: {\tt
  iiop://myhost:1234}. This property supercedes OAIAddr and OAPort.
& string & unset \\
\hline
\verb"OAIAddr" & The Object Adapter Internet Address: IP address on
multi-homed host (this gets encoded in object references).
\begin{itemize}
\item Addresses like 127.0.0.X will only be accessible from the same
machine!
\item If OAIAddr is not set on a multi-homed host it is operating system/JVM
dependant which IP address is selected.
\item If the developer is trying to use callbacks (\textit{not\textit{ bidirectional
GIOP)}} on a multihomed host the client
will also require OAIAddr set as it is acting as a server.
\end{itemize}
  & node & unset \\
\hline
\verb"OAPort" & See OAIAddr above (ignored if OAAddress is set) & port & unset \\
\hline
\verb"org.omg.PortableInterc"
\verb"eptor.ORBInitializerCl"
\verb"ass.standard_init" & Standard portable interceptor. DO NOT
REMOVE. & class &  \\
\hline
\verb"jacorb.net.socket_fact"
\verb"ory" & Sets or defines the socket factory. See section \ref{sec:customSocketFactories} for details.& class & \\
\hline
\verb"jacorb.net.server_sock"
\verb"et_factory" & Sets or defines the server socket factory. See section \ref{sec:customSocketFactories} for details.& class & \\
\hline
\verb"jacorb.net.socket_fact"
\verb"ory.port.min" & Sets the minimum port number that can be used
for an additional supported socket factory. This property is used in
conjunction with the jacorb.net.socket\_factory.port.max
property. These properties enable the factory to traverse firewalls
through a fixed port range  & integer & unset (disabled) \\
\hline
\verb"jacorb.net.socket_fact"
\verb"ory.port.max" & Sets the maximum port number that can be used
for the additional supported socket factory. Refer to
jacorb.net.socket\_factory.port.min above & integer & disabled\\
\hline
\verb"jacorb.net.tcp_listene"
\verb"r" & Defines a listener for TCP connection events. See \ref{eventLogging}.
& string & disabled\\
\hline
\verb"jacorb.enhanced_thread"
\verb"_name" & Temporarily adds connection endpoints and time (in milliseconds)
that the thread started to the Thread name. To be used to correlate running
threads with entries in debug logs. & string & off\\
\hline
\verb"jacorb.avoidIsARemoteC"
\verb"all" & Always attempt to search for local repository ID information to avoid
the cost of a remote call. In most scenarios this is quicker than the remote call.
With some complicated hierarchies it may be quicker to turn this off. & boolean & on\\
\hline
\verb"jacorb.native_char_cod"
\verb"eset" & Overrides the detection from the local environment for the codeset
used to transmit characters. Note that this property is only effective once per
JVM. & string & off\\
\hline
\verb"jacorb.native_wchar_co"
\verb"deset" & Overrides the detection from the local environment for the codeset
used to transmit wide characters. ote that this property is only effective once per
JVM. & string & off\\
\hline
\verb"jacorb.codeset" &
Enabling this will do codeset translation on marshalling. Disabling it will force
JacORB to ignore all codeset component info profiles and to disable translation on
marshalling. & boolean & on\\
\hline
\end{longtable}
\end{small}

\textbf{Note:} The class {\tt org.jacorb.orb.giop.CodeSet} provides a main method to
aid debugging of codeset issues. It will print out the current system encoding values.
If the developer is running under a Unix based system and passes the argument -a it will
also print out the current locale and all known locales.


\begin{small}
\begin{longtable}{|p{5cm}|p{7.5cm}|p{1.5cm}|p{1.5cm}|}
\caption{Logging Configuration}\\
\hline
~ \hfill \textbf {Property} \hfill ~ & ~ \hfill \textbf {Description}
\hfill ~ & ~ \hfill \textbf {Type} \hfill ~ & ~ \hfill
\textbf{Default} ~ \endhead
\hline
\verb"jacorb.orb.print_versi"
\verb"on" & If enabled, the ORB's version number is printed whenever
the ORB is initialized. & boolean & on \\
\hline
\verb"jacorb.log.default."
\verb"verbosity" & Log levels: 0 = off, 1 =
error, 2 = warning, 3 = info, 4 = debug & integer & unset \\
\hline
\verb"jacorb.logfile" & Output destination for diagnostic log file. If
not set, diagnostics are sent to standard error.& filename & unset \\
\hline
\verb"jacorb.logfile.append" & Whether to append to
existing log file or overwrite (if file logging) & boolean & off \\
\hline
\verb"jacorb.log.initializer" & Name of a class to initialize logging
after the configuration has been read & class &
\verb"JdkLogg"
\verb"ingInit"
\verb"ializer" \\
\hline
\verb"jacorb.debug.dump_outg"
\verb"oing_messages" & Hex dump outgoing messages & boolean & off \\
\hline
\verb"jacorb.debug.dump_inco"
\verb"ming_messages" & Hex dump incoming messages & boolean & off \\
\hline
\end{longtable}
\end{small}


\begin{small}
\begin{longtable}{|p{5cm}|p{9cm}|p{2cm}|}
\caption{Name service Configuration}\\
\hline
~ \hfill \textbf {Property} \hfill ~ & ~ \hfill \textbf {Description} \hfill ~ & ~ \hfill \textbf {Type} \hfill ~ \endhead
\hline

\verb"jacorb.naming.purge" & Whether non-active references are purged from name service
when list operation is invoked. Default is off & on or off \\
\hline
\verb"jacorb.naming.noping" & Whether resolve should return references without trying to
ping them to see if they're still alive first. Default is ping (off) &
on or off\\
\hline
\verb"jacorb.naming." & The file where the name server & string\\
\verb"ior_filename" &  drops its IOR (default unset) &   \\

\hline
\end{longtable}
\end{small}


\begin{small}
\begin{longtable}{|p{5cm}|p{9cm}|p{2cm}|}
\caption{POA Configuration}\\
\hline
~ \hfill \textbf {Property} \hfill ~ & ~ \hfill \textbf {Description} \hfill ~ & ~ \hfill \textbf {Type} \hfill ~ \endhead
\hline
\verb"jacorb.poa.monitoring" & Displays a GUI monitoring tool for servers. Default is off. & boolean \\
\hline
\verb"jacorb.poa.thread_pool"
\verb"_max" & Maximum thread pool configuration for request processing & integer \\
\hline
\verb"jacorb.poa.thread_pool"
\verb"_min" & Minimum thread pool configuration for request processing & integer \\
\hline
\verb"jacorb.poa.thread_pool"
\verb"_shared" & If set use shared thread pool between all POAs. Only with
ORB\_CTRL\_MODEL. Default is off. & boolean \\
\hline
\verb"jacorb.poa.thread_prio"
\verb"rity" & If set, request processing threads in the POA will run at this priority. If not set or invalid, MAX\_PRIORITY will be used. Not set by default. & integer \\
\hline
\verb"jacorb.poa.queue_wait" & Specifies whether the POA should block
when the request queue is full (On), or throw TRANSIENT exceptions
(Off). Default is Off. & boolean\\
\hline
\verb"jacorb.poa.queue_max" & The maximum length of the request
queue.  If this length has been reached, and further requests arrive,
jacorb.poa.queue\_wait specifies what to do. Default is 100. & integer \\
\hline
\verb"jacorb.poa.queue_min" & If jacorb.poa.queue\_wait is On, and the
request queue gets full, then the POA blocks until the queue contains
no more than queue\_min requests. Default is 10. & integer \\
\hline

\end{longtable}
\end{small}


\begin{small}
\begin{longtable}{|p{5cm}|p{9cm}|p{2cm}|}
\caption{Implementation Repository Configuration}\\
\hline
~ \hfill \textbf {Property} \hfill ~ & ~ \hfill \textbf {Description} \hfill ~ & ~ \hfill \textbf {Type} \hfill ~ \endhead
\hline
\verb"jacorb.use_imr" & Switch on to contact the Implementation Repository (IR) on every server start-up. Default is off. & boolean \\
\hline
\verb"jacorb.use_imr_endpoin"
\verb"t" & Switch off to prevent writing the IMR address into server IORs. This property is ignored if jacorb.use\_imr = off. Default is off. & boolean \\
\hline
\verb"jacorb.imr.allow_auto_"
\verb"register" & If set to on servers that don't already have an entry on their first call to the IR, will get automatically registered. Otherwise, an UnknownServer exception is thrown. Default is off. & boolean \\
\hline
\verb"jacorb.imr.check_objec"
\verb"t_liveness" & If set on the IR will try to ping every object reference that it is going to return. If the reference is not alive, then TRANSIENT is thrown. Default is off. & boolean \\
\hline
\verb"ORBInitRef.Implementat"
\verb"ionRepository" & The initial reference for the IR. & URL \\
\hline
\verb"jacorb.imr.table_file" & File in which the IR stores data. & file \\
\hline
\verb"jacorb.imr.backup_file" & Backup data file for the IR. & file \\
\hline
\verb"jacorb.imr.ior_file" & File to which the IR writes its IOR. This is usually referred to by the initial reference for the IR (configured above).  & file \\
\hline
\verb"jacorb.imr.timeout" & Time in milliseconds that the implementation will wait for a started server to register. After this timeout is exceeded the IR assumes the server has failed to start. Default is 12000 (2 minutes). & millisec. \\
\hline
\verb"jacorb.imr.no_of_poas" & Initial number of POAs that can be registered with the IR. This is an optimization used to size internal data structures. This value can be exceeded. Default is 100. & integer \\
\hline
\verb"jacorb.imr.no_of_serve"
\verb"rs" & Initial number of servers that can be registered with the IR. This is an optimization used to size internal data structures. This value can be exceeded. Default is 5. & integer \\
\hline
\verb"jacorb.imr.port_number" & Starts the IMR on a fixed port (equivalent to the -p option). & integer \\
\hline
\verb"jacorb.imr.connection_"
\verb"timeout" & Time in milliseconds that the IR waits until a connection from an application client is terminated. Default is 2000. & millisec. \\
\hline
\verb"jacorb.implname" & The implementation name for persistent servers. See \ref{implname}. & name \\
\hline
\verb"jacorb.java_exec" & Command used by the IR to start servers. & command \\
\hline

\end{longtable}
\end{small}


\begin{small}
\begin{longtable}{|p{5cm}|p{9cm}|p{2cm}|}
\caption{Security Configuration}\\
\hline
~ \hfill \textbf {Property} \hfill ~ & ~ \hfill \textbf {Description} \hfill ~ & ~ \hfill \textbf {Type} \hfill ~ \endhead
\hline
\verb"OASSLPort" & The port number used by SSL, will be dynamically assigned by default. & port \\
\hline
\verb"org.omg.PortableInterc"
\verb"eptor.ORBInitializerCl"
\verb"ass.ForwardInit" & Portable interceptor required to support SSL. Not set by default. & class \\
\hline
\verb"jacorb.security.access"
\verb"_decision" & The qualified classname of access decision object. & class \\
\hline
\verb"jacorb.security.princi"
\verb"pal_authenticator" & A list of qualified classnames of principle authenticator objects, separated by commas (no whitespaces.). The first entry (that can be successfully created) will be available through the principal\_authenticator property. & class \\
\hline
\verb"jacorb.ssl.socket_fact"
\verb"ory" & The qualified classname of the SSL socket factory class. See section \ref{sec:customSocketFactories} for details.& class \\
\hline
\verb"jacorb.ssl.server_sock"
\verb"et_factory" & The qualified classname of the SSL server socket factory class. See section \ref{sec:customSocketFactories} for details.& class \\
\hline
\verb"jacorb.security.suppor"
\verb"t_ssl" & Whether SSL security is supported. Default is off. & boolean \\
\hline
\verb"jacorb.security.ssl.cl"
\verb"ient.supported_options" & SSL client supported options - IIOP/SSL parameters (numbers are hex values, without the leading 0x): NoProtection = 1, EstablishTrustInClient = 40, EstablishTrustInTarget = 20, mutual authentication = 60. Default is 0. Please see the programming guide for more explanation. & integer \\
\hline
\verb"jacorb.security.ssl.cl"
\verb"ient.required_options" & SSL client required options (See IIOP/SSL parameters above). Default is 0. & integer \\
\hline
\verb"jacorb.security.ssl.se"
\verb"rver.supported_options" & SSL server supported options (See IIOP/SSL parameters above). Default is 0. & integer \\
\hline
\verb"jacorb.security.ssl.se"
\verb"rver.required_options" & SSL server required options (See IIOP/SSL parameters above). Default is 0. & integer \\
\hline
\verb"jacorb.security.ssl.co"
\verb"rbaloc_ssliop.supporte"
\verb"d_options" & Used in conjunction with jacorb.security.ssl.corbaloc\_ssliop.required\_options. If these properties are set, then two values will be placed in the IOR, "corbaloc:ssliop" and "ssliop". If not set, only EstablishTrustInTarget is used for both supported and required options. & integer \\
\hline
\verb"jacorb.security.ssl.co"
\verb"rbaloc_ssliop.required"
\verb"_options" &  Default is 0. & integer \\
\hline
\verb"jacorb.security.ssl.al"
\verb"ways_open_unsecured_en"
\verb"dpoint" &  Default is FALSE. The secure interoperabilty spec states that targets that require SSL shall not open (or publicise in their IORs) an unsecured listen port. Some ORBs (we're looking at you, MICO) apparently don't like this. Setting this switch to TRUE will override the correct behaviour for interoperability. Attempts to access the unsecured port should be met with a NO\_PERMISSION exception. & boolean \\
\hline
\verb"jacorb.security.keysto"
\verb"re" & The name and location of the keystore. This may be absolute or relative to the home directory. NOTE (for Sun JSSE users): The javax.net.ssl.trustStore [Password] properties doesn't seem to take effect, so you may want to add trusted certificates to normal keystores. In this case, please set the property jacorb.security.jsse.trustees\_from\_ks to on, so trusted certificates are taken from the keystore instead of a dedicated truststore.  & file \\
\hline
\verb"jacorb.security.keysto"
\verb"re_password" & The keystore password. & string \\
\hline
\verb"jacorb.security.keysto"
\verb"re_type" & The SSL keystore type. Defaults to JKS. & string \\
\hline
\verb"jacorb.security.jsse.s"
\verb"erver.key_manager_a"
\verb"lgorithm" & The algorithm used to initialise the SSL socket factories. Defaults to SunX509. Change to IbmX509 for IBM JDKs. & string \\
\hline
\verb"jacorb.security.jsse.s"
\verb"erver.trust_manager_a"
\verb"lgorithm" & The algorithm used to initialise the SSL socket factories. Defaults to SunX509. Change to IbmX509 for IBM JDKs. & string \\
\hline
\verb"jacorb.security.jsse.c"
\verb"lient.key_manager_a"
\verb"lgorithm" & The algorithm used to initialise the SSL socket factories. Defaults to SunX509. Change to IbmX509 for IBM JDKs. & string \\
\hline
\verb"jacorb.security.jsse.c"
\verb"lient.trust_manager_a"
\verb"lgorithm" & The algorithm used to initialise the SSL socket factories. Defaults to SunX509. Change to IbmX509 for IBM JDKs. & string \\
\hline
\verb"jacorb.security.jsse.t"
\verb"rustees_from_ks" & Sun JSSE specific settings: Use the keystore to take trusted certificates from. Default is off. & boolean \\
\hline
\verb"jacorb.security.ssl.se"
\verb"rver.cipher_suites" & A comma-separated list of cipher suite names which must NOT contain whitespaces. See the JSSE documents on how to obtain the correct cipher suite strings. & string \\
\hline
\verb"jacorb.security.ssl.cl"
\verb"ient.cipher_suites" & See jacorb.security.ssl.server.cipher\_suites above. & string \\
\hline
\verb"jacorb.security.random"
\verb"classPlugin" & Classname for secure random plugin. See \ref{secureRandomPlugin} & string \\
\hline
\verb"jacorb.security.ssl.ss"
\verb"l_listener" & Defines a listener for SSL connection events. See \ref{eventLogging}.
& string \\
\hline
\end{longtable}
\end{small}

\subsubsection{Acceptor Exception Event Plugin}
\label{acceptorevent}
This plugin is implemented by {\tt
org.jacorb.orb.listener.AcceptorExceptionListener}.
\begin{small}
\begin{verbatim}
package org.jacorb.orb.listener;
public interface AcceptorExceptionListener extends EventListener
    void exceptionCaught(AcceptorExceptionEvent ae);
\end{verbatim}
\end{small}
The configuration property is
\begin{verbatim}
jacorb.acceptor_exception_listener
\end{verbatim}
If the server listener thread receives an exception while doing the {\tt
ServerSocket.accept()} it will construct a {\tt
org.jacorb.orb.listener.AcceptorExceptionEvent} and notify the configured
implementation. The Event allows the following to be retrieved:
\begin{small}
\begin{verbatim}
    public ORB getORB()
    public Throwable getException()
\end{verbatim}
\end{small}
The default implementation, {\tt
org.jacorb.orb.listener.DefaultAcceptor\-ExceptionListener}, will simply shutdown
the ORB on all Errors and for SSLExceptions that are thrown before any socket
connections have been made. If the developer wishes they may plugin
their own for more fine grained control.

In order to detect whether the exception has been thrown on the first attempt
or any attempt after that the developer may use the following function within
their listener implementation.
\begin{small}
\begin{verbatim}
    public void exceptionCaught(AcceptorExceptionEvent ae) {
    ...
       if (((org.jacorb.orb.iiop.IIOPListener.Acceptor)
                   ae.getSource()).getAcceptorSocketLoop()) {
      ...
\end{verbatim}
\end{small}
{\tt getAcceptorSocketLoop} returns false if the event has been thrown on the
initial loop, or true on any loop after that.

Note that if the default implementation is used it is possible that due to e.g.
an SSLException the listener will fail to accept on the server socket after the
root POA is resolved which means that the ORB will be shutdown. Therefore future
calls on that POA will fail with a 'POA destroyed' message.

\subsection{JacORB Implname and CORBA Objects}
\label{implname}
A JacORB object key consists of {\tt <impl name>/<poa name>/<object oid>}. The lifespan of CORBA objects are defined by the POA policy LifespanPolicyValue.

Transient objects are those whose lifespans are bounded by the process in which they were created. Once a transient object has been destroyed any clients still holding references to those objects should receive a OBJECT\_NOT\_EXIST. This applies even if the transient object is recreated as it is a new object reference. To achieve this JacORB replaces the implname portion of the key with transient data.

Persistent objects are those that may live beyond the lifetime of the process that created them. The implname property should be configured in this case. It should be set to a unique name to to form part of the object identity. If it is not set, an exception will be thrown. This property may be configured in the jacorb.properties (where an example shows it set to StandardImplName) or in the code of the server e.g.
\small{
\begin{verbatim}
    /* create and set properties */
    java.util.Properties props = new java.util.Properties();
    props.setProperty("jacorb.use_imr","on");
    props.setProperty("jacorb.implname","MyName");

    /* init ORB  */
    orb = org.omg.CORBA.ORB.init(args, props);
\end{verbatim}
}

The implname property allows a program to run with a different implementation name so that it will not accept references created by another persistent POA with the same POA name. A common problem is where the developer has two persistent servers running with the same implname and POA names when one tries to contact the other. Rather than calling server x, server y performs local call. This is because there is no way of distinguishing the two servers; the developer should have used different implnames (e.g. UUIDs).

\subsection{Corbaloc Strings}

Normally corbaloc is used to provide a shortcut to refer to CORBA objects. However the stringified key portion corresponds to the octet sequence in the object\_key member of a GIOP Request or LocateRequest header as defined in section 15.4 of CORBA 2.3. Further the key\_string uses the escape conventions described in RFC 2396 to map away from octet values that cannot directly be part of a URL. This means the key string might look like:
\small{
\begin{verbatim}
corbaloc:iiop:10.1.0.4:18000/FooBar/ServiceName/V_3%f1%1c%9b%11%db%b7%e9
         %bdsnQ%ea%85qV_3%f0%1c%9b%11%db%b7%e9%bdsnQ%ea%85TA5%f0%1c%9b%11
         %db%b7%e9%bdsnQ%ea%85
\end{verbatim}
}
With JacORB, for persistent objects, the developer may configure the implname, poa name and object key. This should mean that the corbaloc sequence should be more readable:
\small{
\begin{verbatim}
corbaloc:iiop:10.1.0.4:42811/imr_demo/ImRDemoServerPOA/imr_demo
\end{verbatim}
}
With a transient object the key may look like:
\small{
\begin{verbatim}
corbaloc:iiop:10.1.0.4:42818/2649480905/%00%14%3e45%0d%0b!%10%3e
\end{verbatim}
}
As it is not possible to construct a transient object with a readable key some developers may find it useful to use the objectKeyMap facility within JacORB to refer to their transient objects. Note the objectKey functionality may also be used with persistent objects.

This property provides more readable corbaloc URLs by mapping the actual object key to an arbitrary string. The mapping below would permit clients of a name service to access it using corbaloc::ipaddress:portnum/NameService. The property also accepts the following mappings:
\begin{itemize}
\item IOR, resource, jndi, URL (e.g. file, http)
\end{itemize}
Note that {\tt jacorb.orb.objectKeyMap.name} is configurable both through the jacorb.properties file and through the proprietary function

{\tt ORB::addObjectKey(String name, String)}

Example usage

{\tt jacorb.orb.objectKeyMap.NameService=file:///home/rnc/NameSingleton.ior}

This then allows the corbaloc key portion to simply be 'NameService'.

The JacORB utility {\tt dior} may be used to decode IORs. This has an additional command line option to output a corbaloc representation of an IOR. See chapter \ref{ch:tools}.

\subsection{JacORB Network Event Logging}
\label{eventLogging}

An enhancement has been added to JacORB that allows a developer to monitor TCP and SSL connections.
Note that for both of these implementations full information may only retrieved with a successful connection;
e.g. if the connection could not be established there will be no certificates.

\subsubsection{TCP Monitoring}

To monitor TCP connections a developer should implement the following interface
\begin{small}
\begin{verbatim}
package org.jacorb.orb.listener;
public interface TCPConnectionListener extends EventListener
    void connectionOpened(TCPConnectionEvent e);
    void connectionClosed(TCPConnectionEvent e);
\end{verbatim}
\end{small}
The classname should then be specified in the property
\begin{verbatim}
jacorb.net.tcp_listener
\end{verbatim}

The standard java event interface is followed; the developer's code will receive the
TCPConnectionEvent which allows the following information to be retrieved:
\begin{small}
\begin{verbatim}
    public String getLocalIP()
    public int getLocalPort()
    public String getRemoteIP()
    public int getRemotePort()
\end{verbatim}
\end{small}
Note that the TCPConnectionEvent extends java.util.EventObject and the EventObject.getSource
operation will return the IIOPConnection of the TCP connection.

\subsubsection{SSL Monitoring}

To monitor SSL sessions a developer should implement the following interface
\begin{small}
\begin{verbatim}
package org.jacorb.orb.listener;
public interface SSLSessionListener extends EventListener
    void sessionCreated(SSLSessionEvent e);
    void handshakeException(SSLSessionEvent e);
    void keyException(SSLSessionEvent e);
    void peerUnverifiedException(SSLSessionEvent e);
    void protocolException(SSLSessionEvent e);
    void sslException(SSLSessionEvent e);
\end{verbatim}
\end{small}
The classname should then be specified in the property
\begin{verbatim}
jacorb.security.ssl.ssl_listener
\end{verbatim}

The standard java event interface is followed; the developer's code will receive the
SSLSessionEvent which allows the following information to be retrieved:
\begin{small}
\begin{verbatim}
    public String getLocalIP()
    public int getLocalPort()
    public String getRemoteIP()
    public int getRemotePort()
    public String getRemoteDN()
    public X509Certificate[] getPeerCertificateChain()
\end{verbatim}
\end{small}

Note that getRemoteDN will simply return a concatenated string of the
certificates. For that reason it is deprecated; getPeerCertificateChain should
be used instead as that allows a developer to extract specific information from
the certificate.  In order to detect a succesful handshake the implementation
delegates to the JSSE {\tt javax.net.ssl.HandShakeCompletedListener}. When using
JDK1.3 JSSE the JSSE may not throw for instance a handshakeException but a
sslException. Similar to above, SSLSessionEvent extends java.util.EventObject. The
EventObject.getSource operation will return the source of the HandshakeCompletedEvent.

\subsection{JacORB IORMutator}
\label{eventLogging}

An enhancement has been added to JacORB that allows a developer to alter incoming
and outgoing objects at a very low level within the ORB. While the majority of the
users would not require this ability, it is useful within scenarios where for instance,
a user is running with legacy network elements which have multiple, identical IP
addresses. This allows them to mutate the IORs as shown below.

\textbf{This is a very powerful ability that must be used with caution. As it operates
at the CDRStream level it is easy to break the ORB and cause unpredictable behaviour}

\subsubsection{Adding a Mutator}
The developer should firstly extend the following abstract class.
\begin{small}
\begin{verbatim}
package org.jacorb.orb.IORMutator;
public abstract class IORMutator
    protected org.omg.ETF.Connection connection;

    public abstract IOR mutateIncoming (IOR object);
    public abstract IOR mutateOutgoing (IOR object);
\end{verbatim}
\end{small}
The classname should then be specified in the property
\begin{verbatim}
jacorb.iormutator
\end{verbatim}

The IORMutator class also has a {\tt org.omg.ETF.Connection connection} variable. This
variable will be updated with the current transport information for the respective
streams. Note, altering the information within the transport is undefined. The
mutateIncoming operation will be called for CDRInputStream operations and the
mutateOutgoing for CDROuputStream operations.

\subsection{Using custom socket factories}
\label{sec:customSocketFactories}

You may plug in custom socket factories that'll be used by JacORB to
create sockets and server sockets. Each factory needs to implement a JacORB specific
interface. To make your factory available to JacORB you need to set the appropriate
configuration property to the classname of your custom factory. See the following
sections for the available factories and their details. Please also see the javadoc documentation
of the specified interfaces for the contract your custom factories must adhere to.
For convenience JacORB also offers some abstract base classes that pre-implement some functionality and that you
may choose to subclass.

\subsubsection{socket factory}

This factory is used by JacORB to create an outgoing non-SSL connection.

\begin{description}
\item[property] jacorb.net.socket\_factory
\item[implemented interface] org.jacorb.orb.factory.SocketFactory
\item[base class] org.jacorb.orb.factory.AbstractSocketFactory
\end{description}

\subsubsection{server socket factory}

This factory is used by JacORB to create a server socket for incoming non-SSL connections.

\begin{description}
\item[property] jacorb.net.server\_socket\_factory
\item[implemented interface] org.jacorb.orb.factory.ServerSocketFactory
\item[base class] org.jacorb.orb.factory.AbstractSocketFactory
\end{description}

\subsubsection{SSL socket factory}

This factory is used by JacORB to create an outgoing non-SSL connection.

\begin{description}
\item[property] jacorb.ssl.socket\_factory
\item[implemented interface] org.jacorb.orb.factory.SocketFactory
\end{description}

\subsubsection{SSL server socket factory}

This factory is used by JacORB to create a server socket for incoming SSL connections.

\begin{description}
\item[property] jacorb.ssl.server\_socket\_factory
\item[implemented interface] org.jacorb.orb.factory.ServerSocketFactory
\end{description}

%%% Local Variables:
%%% mode: latex
%%% TeX-master: "../ProgrammingGuide"
%%% End:
